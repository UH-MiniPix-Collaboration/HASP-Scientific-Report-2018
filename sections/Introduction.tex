\section{Mission and Objectives}
\label{sec:Introduction}
\newcommand{\indentitem}{\setlength\itemindent{25pt}}%%%indent costume command
%insert image here of payload on gondola or during flight if possible.
\begin{figure}[H]
	\begin{center}
	\includegraphics[width=50 mm, scale=0.5]{figures/gondola.jpg}
	\caption{Payload on the flightline}
	\label{fig:gondola}
	\end{center}
\end{figure}


%Intro (date, time, and information of general flight.  Weather conditions.  Official flight time.  Float start time and date, along with termination time and date.  Impact location.  Did it all go well?
Braving through inclement weather, SORA 2.0 took off early on September 4, 2018 at 14:03 UTC along with 12 student payloads.  From Ft. Sumner Municipal Airport, the payloads flew for approximately 9 hours.  The flight terminated at 1:31 UTC the same day and then landed shortly after at 2:11 UTC about 60 miles southwest of Mt Graham, Arizona.
%%%%%%%%%%%%%%%%%%%%%%%%%%%%%%%
\begin{table}[H]
\centering
\caption{Flight information from NASA \cite{HASP}}
\label{flight_info}
\resizebox{\textwidth}{!}{%
\begin{tabular}{ll}
FLIGHT NUMBER:&688N \\
LAUNCH TIME:&09/04/2018 14:03:22 UTC \\
LAUNCH LOCATION:& 34.473162N 104.242232W \\
FLOAT START:&16:30:48 UTC \\
TERMINATION:&09/05/2018 01:31:23 UTC \\
FLOAT TIME:&09:00:35 \\
IMPACT:&02:11 UTC \\
IMPACT LOCATION:&32.44816666N 109.5093472W which is 60 miles SW of Mt Graham, Arizona 
\end{tabular}%
}
\end{table}


%Mission statement and objectives:
%SORA 2.0 was a continuation of SORA~\cite{SORA} but this time seeking to . . .
SORA 2.0 was a continuation flight to further develop and build upon the first SORA flight \cite{SORA}.  SORA's first flight in 2017 collected valuable information, yet another mission was necessary to confirm and add to the findings of the first SORA flight.  Once again SORA attempted to collect extremophile bacteria and spores that may reside 36 to 41 kilometers in the upper atmosphere.  For the radiation portion of SORA 2.0, a MiniPIX particle detector inside a custom built casing and coupled with a boron loaded scintillator was also flown.  This was to further study the surrounding radiation and attempt to measure neutrons in order to understand the possible effects on extremophiles.  
%%%%%%%%%%%%%%%%%%%%%%%%%%%%%%%%%%%%
%Past scientific questions:
% Are extremophiles present in the upper atmosphere at altitudes of 36 to 41 km?  If extremophiles are captured, can the SORA payload clean box container prevent sample contamination? Finally, can we collect data accurately enough to effectively study the effects of environmental radiation on extremophile organisms and spores.
 
%New scientific questions for SORA 2.0 from provisional application:
\noindent {\bf Scientific Questions}

The goals and objectives for SORA 2.0 are based on the following scientific questions: 
%
\begin{itemize}
	{\indentitem \item After confirming that microorganisms are present in the upper atmosphere in the first SORA mission, what extremophiles are present in the upper atmosphere at altitudes of 36 to 41 km?}
	{\indentitem \item Can Fluropore membrane filters (\SI{0.22}{\micro\meter}) passively capture microorganisms and spores in the upper atmosphere?}
	{\indentitem \item Are the new background control protocols sufficient to confirm microorganism and spore capture?} 
	{\indentitem \item Coupling a boron loaded scintillator to the MiniPIX, can it detect and measure neutrons? } 
	{\indentitem \item If there is a measured difference in the MiniPIX due to a neutron event, is that due to the scintillator?}
	{\indentitem \item Finally, with a deeper understanding of the MiniPIX after the first SORA mission, can SORA 2.0 collect more data to study cosmic radiation that microoganisms and spores experience on a daily basis? Specifically, can SORA 2.0 obtain useful information about the biological effectiveness of this radiation on bacteria through parameters such as linear energy transfer and dose equivalent?}
\end{itemize}
%

%Did we complete our objectives?  If yes, what were they again (insert here briefly).  
%The main goals for SORA were to once again collect extremophile organisms that reside in the upper atmosphere, study the effects of surrounding radiation on these organisms in the stratosphere and gather data pertaining to the environmental conditions in which these organisms reside~\cite{SORA}.  

%From our provisional application:
%The main goals for SORA are to collect extremophile organisms that reside in the upper atmosphere, study the effects of surrounding radiation on these organisms in the stratosphere and gather data pertaining to the environmental conditions in which these organisms reside [1]. More specically, SORA has two sets of main objectives, along with four additional objectives.

%%%%%%%SORA 2.0 Objectives here.\\
% INSERT CITATIONS HERE SINCE WE ARE TAKING FROM OUR OWN APPLICATION

\noindent {\bf Primary Objectives:}
	\begin{enumerate}
	\item Using a refined astrobiology system, attempt to capture bacteria in the upper atmosphere at approximately \SIrange{30}{41}{\kilo\meter} of altitude.
%	
%
%	The Astrobiology portion has been reviewed and updated - F'E
	\item Conduct RNA analysis on samples
	\item Study the cosmic and background radiation that extremophiles may experience
	\end{enumerate}
%
{\bf Secondary Objectives:}
\begin{enumerate}
\item Couple a boron loaded scintillator to half the MiniPIX detector and attempt to measure neutrons.
\item Develop and simplify a radiation and payload flight control system.
%	\item Determine the polar angle of hits on the detector, compare them to payload orientation information and develop simulations to verify the results.
\item Further testing of the astrobiology hardware in flight and the methodology for collection of microbes in extreme environments at high-altitude. 
\item Improve pre and post-flight decontamination procedures.
\item Implement improved background control procedures. 
%Engineering Objectives
\item Implement a variable shutter time for the MiniPIX based on the flux of particles incident on the detector.
\item Analyze the MiniPIX data in real time and downlink relevant radiation statistics.
\item Implement a redundant data storage mechanism.
\item Test an improved enclosure against impacts and harsh environments.
\item Reduce astrobiology collection apparatus size.

\end{enumerate}
%%%%%%%%%%%%%%%%%%%%%%%%%%%%%%%%%%%%
\subsection{Hypothesis and Objectives}
\label{subsec:Hypothesis and Objectives}
\begin{enumerate}
%These are from provisional application:
%We need to review these, add more and remove some.
	\item Based on the collection results from previous HASP payloads we predict the concentration of cells at an altitude of 36 km will be less than \SI{1000}{cells\per\liter}.
		\begin{enumerate}
			\item Objective: Sample a minimum volumetric amount of air at target altitude for the duration of the float phase (approximately 15 to 18 hours).
			\item Status: Our pump was fully functional for the duration of the flight; thus we were able to sample a minimum volumetric amount of air at the target altitude.		\end{enumerate}
%
	\item Based on control samples and testing before flight, we can compare our final flight results to previous applications.
		\begin{enumerate}
			\item Objective: Quantify and characterize any contamination with our laboratory and payload disinfection procedures.
			\item Objective: Minimize the amount of external contamination before flight with thorough decontamination procedures.
			\item Status: The later portion of the RNA sequencing procedures has been put on hold pending further collaboration with Qiagen field representatives.
		\end{enumerate}
%
	\item Based on previous MiniPIX data, we can compare it to the current flight and observe a stark difference due to capture neutron events.
		\begin{enumerate}
			\item Objective: Couple a boron loaded scintillator to half the MiniPIX detector for the SORA 2.0 flight.
			\item Status: Although the data has been received, it is still under analysis.  The team is currently trying to better understand how the scintillator may have affected the results. 
		\end{enumerate}	
	%\item Objective: After capturing samples, analyze data and compare biological effects to similar genotypes found on Earth's surface. (Future flights)
\end{enumerate}
