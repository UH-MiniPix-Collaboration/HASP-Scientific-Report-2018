\subsection{Radiation Simulations}
\label{sec:MiniPIX Simulations}

In order to gain an understanding of the data obtained by the MiniPIX, cosmic ray shower simulations will be conducted. The simulations will be of the payload's ascent from Fort Sumner's elevation to float altitude. The software to be used is known as the Dynamic Atmospheric Shower Tracking Interactive Model Application (DYASTIMA) \cite{dyastima}, which utilizes the Geant4 toolkit \cite{geant4}. DYASTIMA provides a wide range of parameters, allowing the user to personalize the conditions for the simulated cosmic ray showers. The parameters to be used are outlined in Table~\ref{tab:dyastima}.

\begin{table}[!h]
  \begin{center}
    \caption{Parameters used for DYASTIMA simulations.} 
    \label{tab:dyastima}
    \bigskip
    \begin{tabular}{|l|l|}
      \hline
      \multicolumn{1}{|l|}{\bf Parameter} &
      \multicolumn{1}{l|}{\bf Value} \\
      \hline
      Planet Radius & \SI{6371.393}{\kilo\meter} \\ \hline
      Simulation Area Width & \SI{800}{\kilo\meter} \\ \hline
      Geometry Model & SPHERE \\ \hline
      Surface Type & NONE \\ \hline
      Air Density Change & 5\% \\ \hline
      Physics List & FTEP\_BERT\_HP \\ \hline
      Range Cut & \SI{1.00}{\meter} \\ \hline
      North Magnetic Field & \SI{22926.6}{\nano\tesla} \\ \hline
      East Magnetic Field & \SI{2993.8}{\nano\tesla} \\ \hline
      Vertical Magnetic Field & \SI{43407.4}{\nano\tesla} \\ \hline
      Surface Pressure & \SI{1024}{\milli\bar} \\ \hline
      Surface Gravity & \SI{9.81}{\meter\per\second\second} \\ \hline
      Spectrum Altitude & \SI{32000}{\meter} \\ \hline
    \end{tabular}
  \end{center}
\end{table}

The values for magnetic field were obtained using a data retrieval tool created by the United States National Oceanic and Atmospheric Administration~\cite{magnetictool} by inputting the exact time and location of the flight's launch. Additionally, DYASTIMA requires the user to input the spectrum list, which requires the particles and various possible energies the particle can possess. 

%74.83 km
