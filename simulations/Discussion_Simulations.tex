\subsection{MiniPIX Simulations}
\label{sec:MiniPIX Simulations Discussion}

\begin{table}[!h]
  \begin{center}
    \caption{Parameters to be used for DYASTIMA simulations.} 
    \label{tab:dyastima}
    \bigskip
    \begin{tabular}{|l|r|}
      \hline
      {\bf Parameter} &
      \multicolumn{1}{c|}{\bf Value} \\
      \hline
      Planet Radius & \SI{6371.393}{\kilo\meter} \\ \hline
      Simulation Area Width & \SI{800}{\kilo\meter} \\ \hline
      Geometry Model & SPHERE \\ \hline
      Surface Type & NONE \\ \hline
      Air Density Change & 5\% \\ \hline
      Physics List & FTEP\_BERT\_HP \\ \hline
      Range Cut & \SI{1.00}{\meter} \\ \hline
      North Magnetic Field & \SI{22926.6}{\nano\tesla} \\ \hline
      East Magnetic Field & \SI{2993.8}{\nano\tesla} \\ \hline
      Vertical Magnetic Field & \SI{43407.4}{\nano\tesla} \\ \hline
      Surface Pressure & \SI{1024}{\milli\bar} \\ \hline
      Surface Gravity & \SI{9.81}{\meter\per\second\second} \\ \hline
      Spectrum Altitude & \SI{32000}{\meter} \\ \hline
    \end{tabular}
  \end{center}
\end{table}

Although the simulations have not yet been performed, the process will be carried out as follows. The analysis performed on the simulation results will search for only our four types of radiation, which are as follows: alpha, electron, gamma, and muon. Once the data has been formatted, graphs will be constructed similar to those produced for the MiniPIX data in hopes of finding trends between the two. A high similarity between the two data sets means the MiniPIX data is composed more of the four types of interest. Additionally, the hits recorded in the simulations can be roughly categorized between tracks and blobs by utilizing the simulation's recorded zenith angle and energy. This can be accomplished by setting parameters such as a range of angles for each of the two types. This produces a rough estimation of how many of the hits were tracks and how many were blobs. 

The simulations will be of the payload's ascent from Fort Sumner's elevation to float altitude. The software to be used is known as the Dynamic Atmospheric Shower Tracking Interactive Model Application (DYASTIMA)~\cite{dyastima}, which utilizes the Geant4 toolkit \cite{geant4}. DYASTIMA provides a wide range of parameters, allowing the user to personalize the conditions for the simulated cosmic ray showers. The parameters to be used are outlined in Table~\ref{tab:dyastima}.

