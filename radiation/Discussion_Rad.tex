\subsection{Radiation}
\label{sec:Radiation Results Discussion}
By and large the most significant changes in the radiation system for SORA were technical in nature. This iteration of SORA saw vast improvements in the collection and analysis system. Most significant among these improvements were the ability to handle multiple detectors on a single Raspberry Pi, the ability to analyze data in real time and the ability to configure device parameters such as the frame rate in real time via uplink commands.

The 2018 flight also provided an opportunity to compare data in similar conditions to our 2017 flight. While the composition of the data looked fairly similar in terms of LET distributions, the most striking difference observed was that of the detector count rate. The 2018 data showed a net increase of approximately 1-2 counts per second on average. This increase could potentially be explained by the addition of the scintillator coupled to the detector surface or simply by variation in the flux of cosmic radiation in the atmosphere.
