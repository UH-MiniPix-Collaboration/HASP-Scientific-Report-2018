\subsection{Radiation}
\label{sec:Radiation Results Discussion}
By and large the most significant changes in the radiation system for SORA were technical in nature. This iteration of SORA saw vast improvements in the collection and analysis system. Most significant among these improvements were the ability to handle multiple detectors on a single RPI, the ability to analyze data in real time and the ability to configure device parameters such as the frame rate in real time via uplink commands.

The 2018 flight also provided an opportunity to compare data in similar conditions to our 2017 flight. While the composition of the data looked fairly similar in terms of LET distributions, the most striking difference observed was that of the detector count rate. The 2018 data showed a net increase of approximately 1-2 counts per second on average. This increase could potentially be explained by the addition of the scintillator coupled to the detector surface or simply by variation in the flux of cosmic radiation in the atmosphere.

The count difference provides a good platform for future missions during which the neutrons can be more closely examined as the solar cycle develops. Hathaway \cite{SolarCycle} has shown the direct relationship between the development of the solar cycle and the average neutron counts. By maintaining a similar detection technique for several years, the data should display a trend similar to that observed by Hathaway. Additionally, a larger array of MiniPIX devices would likely be required, as the active surface area of a single MiniPIX is rather small. In order to make a confident determination in a trend, a larger active surface area would be required to offset the natural small fluctuation in neutron density.
