\subsection{Cosmic Radiation Methods}
\label{sec:Rad-Methods}
The output format of TimePIX devices provides a great deal of flexibility in terms of data analysis. The data can provide both a simple absorbed dose from ionizing radiation as well as insight into more detailed dosimetric endpoints like dose equivalent. Additionally a clustering procedure can be performed to analyze individual particle interactions in the detector based on their morphological properties.

 Since energy deposition can be determined on a per pixel basis, absorbed energy is calculated via a simple summation across all pixels in a frame of data after applying a calibration procedure \cite{calib}. With knowledge of the mass of the detector volume you can then calculate a dose by dividing the deposted energy by the detector mass. Thus a simple dose in silicon is calculated as



\[D_{Si} = \frac{E}{M_{d}}\]



To analyze the data on a per track basis first a clustering procedure has to be performed to seperate the data in a frame into individual tracks. This can be accomplished by applying a simple algorithm used in image processing called Flood-Fill. Whereby contiguous nonzero pixels are recursively filled and grouped together into clusters. Then, after calculating a minimum aread bounding box and linear least squares fit a track length is calculated by determining where the least square fit line intersects with the bounding box. Additionally, an azimuth angle can be determined relative to the detector orientation and a density can be calculated by dividing the number of pixels in a cluster by the area of the bounding box. A simple visual example of these calculations is shown in Figure \ref{track_analysis}.

\begin{figure}[H]
	\begin{center}
	\includegraphics[width=0.8\textwidth]{figures/density.png}
	\caption{Track parameter calculation.}
	\label{track_analysis}
	\end{center}
\end{figure}

\begin{figure}[H]
	\begin{center}
	\includegraphics[width=0.8\textwidth]{figures/cluster_types.png}
	\caption{Possible physical origins of cluster morphology.}
	\label{tracks}
	\end{center}
\end{figure}
