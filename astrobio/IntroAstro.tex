\subsection{Astrobiology}
\label{sec:Astrobiology-Background}
Extremophiles are microorganisms that thrive in physically and/or chemically extreme conditions in which most life cannot survive. These organisms and microbes have been found everywhere, from deep underwater volcano vents to buried ice lakes in Antarctica (1). 
%Antarctica [1] doesn't work, needs proper citation
Fungi and bacterial spores have also been found in the stratosphere. Today, the most common altitudes for organism and microbe collection in the atmosphere are in the range of approximately 10 km to 20 km above Earth’s surface. Very little data exists on microbiological samples captured in the stratosphere. 
%stratosphere [1] needs proper citation, NOT working in the compiler
Conditions at altitudes of 30km to 40km are extreme in temperature, pressure and radiation exposure. Arguably, each successful collection expedition, of at least 30 km into the upper atmosphere, provides information that can be useful in determining what life forms can exist inside and outside of Earth’s biosphere. Additionally, RNA analysis of the organisms and microbes can provide useful insight pertaining to their ability to survive in an environment with elevated levels of radiation.  

This year our experiment focused on designing a more compact collection apparatus and refining our sanitation procedures for preflight assembly, post flight disassembly, and RNA sequencing preparation. The samples we collected play an important role in expanding our knowledge about Earth’s biosphere. Future studies could produce meaningful contributions to the fields of gene therapy, RNA interface, and cosmic shielding; and provide valuable insight about how life can be distributed on Earth, and ultimately, through outer-space.

