\subsection{Astrobiology}
\label{sec:Astrobiology-Background}
Extremophiles thrive in physically and/or chemically extreme conditions, which are detrimental to most of life on Earth as we know it. These organisms and microbes have been found everywhere, from deep underwater volcano vents to buried ice lakes in Antarctica [1]. Fungi and bacterial spores have also been found in the stratosphere. Today, the most common altitudes for bacterial collection in the atmosphere are in the range of approximately 10km to 20km above Earth’s surface. As illustrated in the table below, very little data exists on microbiological samples captured in the stratosphere [1]. Conditions at altitudes of 30km to 40km are extreme in temperature, pressure and radiation exposure. Arguably, each successful collection expedition, of at least 30km into the upper atmosphere, provides information that can be useful in determining what life forms can exist inside and outside of Earth’s biosphere. Additionally, RNA analysis of the organisms and microbes can provide useful insight pertaining to their ability to survive in an environment with increased radiation exposure.  

Our experiment was an attempt to further develop our technique for capturing microorganisms in the upper atmosphere, as demonstrated during our 2017 flight [1]; which was inspired by the LSU HASP 2011, 2012, and 2013 flights, and from research conducted by D.R. Canales [1]. The samples we collected are an important part to expanding our understanding of Earth’s biosphere. Further studies could provide more insight on how life can be distributed on Earth, and ultimately, through outer-space.

