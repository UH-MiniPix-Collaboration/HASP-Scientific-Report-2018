\subsection{Astrobiology Methods}
\label{sec:Astrobiology Methods}
<<<<<<< HEAD
\subsection{ Flight Preparations }
 The clean box, the collection containers and the tubing to the pump were all autoclaved to sterilize the components. 70\% ethanol solution is circulated through the pump to sterilize the internals. Everything used to assemble the system is then sprayed down with the ethanol solution. The  15 \% glycerol solution was poured into each container, the lid was sealed with epoxy
 and the tubing was inserted and epoxy sealed into the container lids. Each lid had two holes, one that led to the inside of the clean box to allow for pressure to be released from inside the container and outgassed  through the valve embedded in the box, while the other hole passed through the clean box lid to allow the
 tubing to connect to the pump or solenoid only in the case of the control tubing. The filters used for the background samples were then attached to the walls of the payload. The tubing is then connected to the cylinder that controls the lid which exposes the tubing. The payload remained closed and sterilized until the time of flight at Fort Sumner. The servo lid was removed prior to flight to expose the tubing.  
=======
The payload was successfully retrieved on September 6, 2018. The intact payload was shipped to the University of Houston and received on September 12, 2018. The clean box and background samples were removed and placed in cold storage at -20◦C. All equipment used in the filtration process was either autoclaved or taken from previously unopened sanitized packaging. The autoclaved, pre-sanitized items and the clean box were washed in a 70 % ethanol solution before they were placed inside a SterilGARD e3 Class II Biological Safety Cabinet. 
The control sample collection solution was vacuum filtered through a Fluropore membrane filter (13 mm; 0.22 micron) to collect specimens on the filter surface. Due to a lower glycerol volume, the experimental sample collection container was thoroughly swabbed with a Fluropore membrane filter. All filters were processed using the DNeasy PowerWater Kit by QIAGEN in preparation for 16S ribosomal RNA sequencing. All post flight sanitation and sample and control filtration procedures took place under the supervision of Professor Donna Pattison from the Department of Biology and Biochemistry at The University of Houston.
>>>>>>> d473a5d68e5792203606cca161e0c569656da51c
