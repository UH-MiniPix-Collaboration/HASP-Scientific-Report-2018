%\begin{Document}
\subsection{Astrobiology Methods}
\label{sec:Astrobiology Methods}
\subsection{ Flight Preparations }
 The clean box, the collection containers and the tubing to the pump were all autoclaved to sterilize the components. 70\% ethanol solution was circulated through the pump to sterilize the internals. Everything used to assemble the system was then sprayed down with the ethanol solution before being placed inside a SterilGARD e3 Class II Biological Safety Cabinet. The 15\% glycerol solution was pipetted into each container, the lid was sealed with epoxy and the tubing was inserted and epoxy sealed into the container lids. Each lid had two holes, one that led to the inside of the clean box to allow for pressure to be released from inside the container and out-gassed  through the valve embedded in the box, while the other hole passed through the clean box lid to allow the tubing to connect to the pump or solenoid only in the case of the control tubing.The lid to the clean box was sealed with epoxy, the tube from the control container was clamped to the dedicated control solenoid, while the sample collecting tube was passed through the other solenoid and connected to the pump. The tubing connecting the clean box to the pump and control solenoid was secured with epoxy and gasket maker to secure the tubing in place. The solenoid clamping tubing to the experimental container remained closed until float conditions were achieved. Fluropore membrane filters (13mm;22 micron) were used to collect control samples in the labs where work was conducted and on the inside walls of the payload. 
 
 The payload was successfully retrieved on September 6, 2018. The intact payload was shipped to the University of Houston and received on September 12, 2018. The clean box and background samples were removed and placed in cold storage at -20 degrees Celsius.

 All equipment used in the filtration process was either autoclaved or taken from previously unopened sanitized packaging. The autoclaved, pre-sanitized items and the clean box were washed in a 70\% ethanol solution before they were placed inside the Biological Safety Cabinet previously mentioned. 
 
The control sample collection solution was vacuum filtered through a Fluropore membrane filter (13 mm; 0.22 micron) to collect specimens on the filter surface. The experimental sample collection container was thoroughly swabbed with a Fluropore membrane filter as the glycerol coated the inside of the container and the full sample could not be pipetted out of the container to pass through the vacuum and onto the filter. All filters were processed using the DNeasy PowerWater Kit by QIAGEN \cite{PowerWaterKit} in preparation for 16S ribosomal RNA sequencing.   









%\end{Document}
