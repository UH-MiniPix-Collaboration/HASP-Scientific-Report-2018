\subsection{Astrobiology}
\label{sec:Astrobiology-Results}

One goal of this project was to confirm our results from last year. To that end we selected the same 926wF (“AAACTYAAAKGAATTGRCGG”) and 1392R (“ACGGGCGGTGTGTRC***”) primers for the sequencing procedure \cite{SORA}. These primers are designed to amplify 16S RNA from bacterial, archaeal and eukaryotic “universal” samples. The samples were sent to the University of Houston Seq-N-Edit Core \cite{Seq-N-Edit Core} and processed using the QIAseq 16S/ITS protocols \cite{QIAseq}. The samples were quantified using a Qubit Fluorometer; the results yielded 28 of the \SI{25}{\pico\gram\per\micro\liter}-minimum in the experimental sample only. DNA was not detectable in the control samples. The control samples were pooled. Both the pooled control samples and the experimental sample were concentrated using a bead extraction technique. The yield was \SI{248}{\pico\gram\per\micro\liter} in the experimental sample and \SI{141}{\pico\gram\per\micro\liter} in the pooled control background sample. Finally, a midpoint quality control check was  performed where it was determined that, while DNA was detectable, the integrity was too low to merit proceeding to the next phase involving several thousand dollars’ worth of materials. 