\subsection{Astrobiology}
\label{sec:Astrobiology Results Discussion}
We were unable to obtain results from this year’s samples. In our 2017 experiment, we had two samples for processing and it cost our team nearly \$1,500 for RNA sequencing and analysis. This year we had almost a dozen samples, including our background samples, and could not afford to send them to RTL Genomics again. We were fortunate to partner with the Seq-N-Edit Core at the University of Houston, who sponsored the materials and researcher hours related to the DNA processing and preparation for 16S ribosomal RNA sequencing. A key difference in the DNA extraction step was the use of the Qiagen DNeasy Powerwater kit rather than the MoBio PowerWater kit \cite{SORA}. The Qiagen kit is a new release and may not be as robust as the MoBio kit. A field representative from QIAGEN is scheduled to meet with Ian Wilson, in the UH Seq-N-Edit Core, during the week of December 3rd 2018 to discuss the kit and our results. Although we do not have definitive results at this time, we are hopeful that the field representative will provide additional insight on how best to tailor the DNA washing kit protocols for samples types such as ours going forward. It was suggested that we optimize our own set of protocols by collecting multiple ground samples and using the kit with various DNA extraction protocols until we find an optimal method. We have also decided to pursue an altogether different mode of collection for next year in hopes of increasing the number of replicates and the surface area available for collection. If we can obtain a large enough sample-size we hope to culture the organisms collected in future experiments, in addition to sequencing the 16S rRNA. There is much to be learned from living cells that can survive under the float conditions to which our payload was subjected. If our results can be repeated and confirmed, it would also demonstrate that there are more organisms that are able to survive under our payloads flight conditions than previously documented and provide additional confirmation as to the existence of the currently documented organisms.
%We were unable to obtain results from this year’s samples.  In our 2017 experiment, we had two samples for processing and it cost our team nearly 1,500 for RNA sequencing and analysis. This year we had almost a dozen samples, including our background samples, and could not afford to send them to RTL Genomics again. We were fortunate to partner with the Seq-N-Edit Core at the University of Houston, who sponsored the materials and researcher hours related to the DNA processing and preparation for 16S ribosomal RNA sequencing.  A key difference in the DNA extraction step was the use of the Qiagen DNeasy Powerwater kit rather than the MoBio PowerWater kit.  The Qiagen kit is a new release and may not be as robust as the MoBio kit. A field representative from QIAGEN is scheduled to meet with Ian Wilson in the UH Seq-N-Edit Core during the week of December 3rd 2018 to discuss the kit and our results.Although we do not have definitive results at this time, we are hopeful that the field representative will provide additional insight on how best to tailor the DNA washing kit protocols for samples types such as ours going forward. It was suggested that we optimize our own set of protocols by collecting multiple ground samples and using the kit with various DNA extraction protocols until we find an optimal method.  We have also decided to pursue an altogether different mode of collection for next year in hopes of increasing the number of replicates and the surface area available for collection . If we can obtain a large enough sample-size we hope to culture the organisms collected in future experiments, in addition to sequencing the 16S rRNA. There is much to be learned from living cells that can survive under the float conditions to which our payload was subjected. If our results can be repeated and confirmed, it would also demonstrate that there are more organisms that are able to survive under our payloads flight conditions than previously documented and provide additional confirmation as to the existence of the currently documented organisms.
