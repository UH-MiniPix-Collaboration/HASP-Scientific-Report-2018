\subsection{Astrobiology}
\label{sec:Astrobiology Results Discussion}
%We need to go into full detail how changing the payload (removing a pump and half the clean box) may have affected our mission.  This is critical, I believe, since we were only able to see if we captured bacteria, but it limited us to culture it.

Unfortunately, all the information past the Archaea Kingdom classification returned unclassified; which is indicative of missing taxonomic information, at a particular level, retrieved from the National Center for Biotechnology Information (NCBI). For example, if the best match in the RTL Genomic database is classified in NCBI down to Genus, then the RTL Genomic database will label the Species as “Unclassified”. 


There are also instances of “Unknown” and “No Hit” that appear in the results. An “Unknown” classification is an instance where the RTL Genomic database is unable to determine the taxonomic classification at a level with a confidence of at least 0.51 or greater. “No Hit” is assigned for three main reasons. First, when an organism’s sequence is missing from the RTL Genomics database, it can take several months to update the database with new sequencing information from NCBI, European Molecular Biology Laboratory (EMBL) and DNA Data Bank of Japan (DDBL). Additionally, sequences in NCBI, EMBL, and DDBL might be too short or lacking in taxonomic information and are therefore excluded from the RTL Genomic database. Second, an organism’s sequence may not be in any of the major databases, because researchers haven’t finished with updating tasks or the organism may be a new species. The ribosomal RNA synthesis is a great option only if the samples collected are a close enough match to the DNA markers selected to generate a positive hit, otherwise we have no way of knowing if we collected samples representative of a new species. If that were the case, then the only way to test for organisms would be to attempt to culture the samples in conjunction with sequencing. Finally, the sequence may be of low quality and fail to identify as any type of organism. 


There are twenty-nine instances of species classifications with a confidence score of 0.51 or greater in the control sample that are absent in the experimental sample. This may be attributed to the exposure to environmental/float conditions; the control container may have preserved all the bacteria accumulated before the final sealing procedures, but the bacteria did not survive in the experimental sample that was exposed to the upper atmospheric environment. There is also the question of the effects of Archaeal metabolism on its immediate surroundings; there is still a great deal that is yet unknown about Archaeal organisms. The metabolic by-products of any organisms accumulated in the experimental sample may have killed any species accumulated prior to launch.  If the abundance was low enough, these species may not have amplified to a detectable level in the PCR steps.




There are twenty-one instances of species classifications with a confidence score of 0.51 or greater in the experimental sample that are absent in the control sample. It is possible that these organisms were collected around the rim of the tubing outside the payload after it was cut and then pumped into the experimental container once float conditions were reached. However, it is unlikely that all of the organisms were collected this way, since the tubing was drenched in alcohol prior to flight and the potential outside bacteria would have been at risk for hypoxia, hypobaria and other environmental stressors related to reaching float altitude. The uncertainty about these classified species can be addressed in a future design that affords protection to the very outer tubing until float conditions are reached.  


Most convincing is the presence of Archaea in the experimental sample only. Archaea were previously classified as extremophiles only, which leaves room for the possibility that our “Unclassified” Archaea, with a confidence score of at least 0.51, are in fact extremophiles collected from the stratosphere. It would make for a more convincing argument had we replicated our collection and control samples in triplicate; this issue will be addressed in future endeavors. 


While there is room to doubt that every organism absent in the control sample, yet present in the experimental sample, was collected at float altitude, it can be argued with some degree of confidence that it is probable that at least the Archaeal organisms were collected as intended. Culturing the organisms collected in future experiments, in addition to sequencing the 16S rRNA, will provide more concise evidence as to the identity and nature of the organisms collected.


There is much to be learned from living cells that can survive under the float conditions to which our payload was subjected. If our results can be repeated and confirmed, it would also demonstrate that there are more organisms that are able to survive under our payload’s flight conditions than previously documented. 


