\documentclass[aps,superscriptaddress,floatfix,nofootinbib,showpacs,amsmath,amssymb,altaffilletter,floatfix,onecolumn]{revtex4-1}
%add "rmp" within document class to make it double column and "twocolumn"
%---

\input{def.tex}
\usepackage{fancyhdr}
\usepackage{enumitem}
\usepackage{graphicx}
\usepackage{graphicx}
\usepackage{wrapfig}
\usepackage[title]{appendix}
\usepackage{float}
%\linespread{0.9}
\fancyhf{}
\renewcommand{\headrulewidth}{0pt}
\rfoot{\thepage}
\pagestyle{fancy}
%\linespread{0}
\renewcommand{\thepage}{}
\renewcommand{\thepage}{\arabic{page}}
\renewcommand\thesection{\arabic{section}}
\renewcommand\thesubsection{\thesection.\arabic{subsection}}
\renewcommand\thesubsubsection{\thesubsection.\arabic{subsubsection}}
\parskip = 6pt %changes spacing between paragraphs
%\nolinenumbers
\makeatletter
\def\p@subsection{}
\makeatother
\makeatletter
\def\p@subsubsection{}
\makeatother
%---
\begin{document}
%---
\title{SORA: Stratospheric Organism and Radiation Analyzer}

\begin{abstract}
\begin{center}
{\bf Abstract}
%from our provisional application:
The SORA 2.0 payload will again sample for the existence of microorganisms and bacterial spores in the upper atmosphere. This mission will build upon the first SORA mission in 2017, and help
confirm previous findings using a further developed capture system. Furthermore, the payload will study different aspects of the surrounding environment such as radiation exposure, temperature,
pressure and humidity. The payload has three main scientific objectives. First, build upon and further develop a novel system that will isolate surrounding air and sample for cells. Second, an
on-board MiniPIX USB silicon sensor will analyze exposure to cosmic radiation that microorganisms may encounter. Finally, a mature version of RESU (Real-time Environmental Sensing Unit) will monitor the environmental conditions such as temperature, pressure, and humidity. The payload design will take advantage of additive manufacturing and hobby electronics in its construction to provide an accessible basis for future missions and explore the bounds of the technology available.

\end{center}
\newpage %Breaks page for the Table of Contents.
\end{abstract}
\newcommand{\Physics}{Department of Physics, University of Houston, Houston, TX, 77024, USA}
\newcommand{\CS}{Department of Computer Science, University of Houston, Houston, TX, 77024, USA}
\newcommand{\Biology}{Department of Biology and Biochemistry, University of Houston, Houston, TX, 77024, USA}
%--- Add other authors in the order they should appear

\author{S.~A.~Garcia~Morelos}\affiliation{\Physics}
\author{F.~Brooks}\affiliation{\Physics}
\author{S.~Oliver}\affiliation{\Physics}
\author{A.~Walker}\affiliation{\CS}
\author{K.~D.~Portillo}\affiliation{\CS}
\author{R.~B.~Masek}\affiliation{\Physics}
\author{J.~Patel}\affiliation{\Physics}
\author{S.~George}\affiliation{\Physics}
\author{I.~Wilson}\affiliation{\Biology}
\author{D.~Pattison}\affiliation{\Biology}
\author{P.~Gunaratne}\affiliation{\Biology}
\author{A.~L.~Renshaw}\affiliation{\Physics}
\setlength{\parindent}{1em}
\setdefaultleftmargin{1em}{1em}{}{}{}{}
%---
\setcounter{page}{0}\thispagestyle{empty}
%---
\maketitle
\onecolumngrid
\setcounter{tocdepth}{2}
\setcounter{page}{0}\thispagestyle{empty}
\tableofcontents
\setcounter{page}{0}\thispagestyle{empty}
\newpage
%---
\onecolumngrid

%Section: Introduction
\subimport{sections/}{Introduction.tex}
\subimport{astrobio/}{IntroAstro.tex}
\subimport{radiation/}{IntroRad.tex}

%Section: SORA Hardware Description
\subimport{sections/}{Hardware.tex}
\subimport{astrobio/}{Design_Astro.tex}
\subimport{radiation/}{Design_RAD.tex}

%Section: Methods
\subimport{sections/}{Methods.tex}
\subimport{astrobio/}{Methods_Astro.tex}
\subimport{radiation/}{Methods_RAD.tex}

%Results and Analysis
\subimport{sections/}{Results.tex} 
\subimport{astrobio/}{Results_Astro.tex} 
\subimport{radiation/}{Results_Rad.tex} 


%Discussion
\subimport{astrobio/}{Discussion_Astro.tex}
\subimport{radiation/}{Discussion_Rad.tex}
\subimport{simulations/}{Discussion_Simulations.tex}

%Conclusion
\subimport{sections/}{Conclusion.tex} 
\newpage

%Appendix
\subimport{sections/}{Appendix.tex}
\newpage

%References
\subimport{bib/}{bib.tex}%Bibliography file

\clearpage
\bibliographystyle{SORA}
\end{document}
